% !TEX encode = UTF-8
% !TEX program = xelatex
\documentclass[answer]{USTBExam}
% \documentclass[noanswer]{USTBExam}

\usepackage{HTNotes-math,esint,extarrows}

% 打补丁
\usepackage{etoolbox}
% 实现连续编号
\patchcmd{\makepart}{\setcounter{problem}{0}}{}{}{}
% 实现选择题括号改为方括号
\patchcmd{\pickout}{(\makebox[1.5em]{\answer{#1}})}{【\makebox[1.5em]{\answer{#1}}】}{}{}
\patchcmd{\pickin}{(\makebox[1.5em]{\answer{#1}})}{【\makebox[1.5em]{\answer{#1}}】}{}{}
% 实现试卷替换为期中试卷
\patchcmd{\makehead}{试卷}{期中试卷}{}{}

% =================================================
%       PDF信息
% =================================================
% PDF信息里的标题栏
\title{2019-2020学年第一学期微积分AI期中模拟试卷}
% PDF信息里的作者栏
\author{黄腾}
% PDF信息里的主题
\Subject{nothing}
% PDF信息里的关键词
\Keywords{高数; 微积分; 期末}

% =================================================
%       试卷头信息
% =================================================
% 试卷头里的年份
\Year{2019--2020}
% 试卷头里的学期
\Semester{一}
% 试卷头里的课程
\Course{微积分AI}
% 试卷头里的类型,如A/B/模拟等
\Type{模拟}
% 试卷头计分表中大题的数目
\TotalPart{4}

\begin{document}

% 生成试卷表头
\makehead

\makepart{单项选择题}{本题共10小题,每题4分,满分40分}

\begin{problem}
  
  \pickout{}
  \options{}
    {}
    {}
    {}
\end{problem}

\begin{problem}
  
  \pickout{}
  \options{}
    {}
    {}
    {}
\end{problem}

\begin{problem}
  
  \pickout{}
  \options{}
    {}
    {}
    {}
\end{problem}

\begin{problem}
  
  \pickout{}
  \options{}
    {}
    {}
    {}
\end{problem}

\begin{problem}
  
  \pickout{}
  \options{}
    {}
    {}
    {}
\end{problem}

\begin{problem}
  
  \pickout{}
  \options{}
    {}
    {}
    {}
\end{problem}

\makepart{填空题}{本题共10小题,每题4分,满分40分}

\begin{problem}
  
  \fillin{}.
\end{problem}

\begin{problem}
  
  \fillin{}.
\end{problem}

\begin{problem}
  
  \fillin{}.
\end{problem}

\begin{problem}

  \fillin{}.
\end{problem}

\begin{problem}
  
  \fillin{}.
\end{problem}

\begin{problem}
  
  \fillin{}.
\end{problem}

\makepart{计算题}{本题共2小题,每题6分,满分12分}

\begin{problem}
  
\end{problem}

\bigskip

\begin{solution}
  
\end{solution}

\begin{problem}
  
\end{problem}

\bigskip

\begin{solution}
  
\end{solution}

\begin{problem}
  
\end{problem}

\bigskip

\begin{solution}
  
\end{solution}

\begin{problem}
  
\end{problem}

\begin{solution}
  
\end{solution}

\makepart{证明题}{本题满分8分}

\begin{problem}
  
\end{problem}

\bigskip

\begin{solution}
  
\end{solution}

\begin{problem}
  
\end{problem}

\bigskip

\begin{solution}
  
\end{solution}

\end{document}
